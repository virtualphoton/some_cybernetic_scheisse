% ОБЯЗАТЕЛЬНО ИМЕННО ТАКОЙ documentclass!
% (Основной кегль = 14pt, поэтому необходим extsizes)
% Формат, разумеется, А4
% article потому что стандарт не подразумевает разделов
% Глава = section, Параграф = subsection
% (понятия "глава" и "параграф" из документа, описывающего диплом)
\documentclass[a4paper,12pt]{extarticle}

% Подключаем главный пакет со всем необходимым
\usepackage{spbudiploma_tempora}

% Пакеты по желанию (самые распространенные)
% Хитрые мат. символы
\usepackage{euscript}
% Таблицы
\usepackage{longtable}
\usepackage{makecell}
% Картинки (можно встявлять даже pdf)
\usepackage[pdftex]{graphicx}

\usepackage{amsthm,amssymb, amsmath}
\usepackage{textcomp}


\begin{document}

% Титульник в файле titlepage.tex
\input{titlepage_example.tex}

% Содержание
\tableofcontents
\pagebreak

\specialsection{Введение}

Эта работа является частью ВКР, целью которой является построение оптимального управления коллективом роботов для игры в шахматы. Основной особенностью является то, что рабочее пространство одного робота не покрывает полностью игровую доску, из-за чего для исполнения некоторых ходов требуется кооперация нескольких роботов. Очевидным алгоритмом является строго последовательное выполнение: робот ждет своего хода в некотором начальном положении, перемещает фигуру, а затем снова перемещается в начальное положение. Это гарантирует, что роботы не столкнутся, однако является неоптимальным по времени: движения можно скоординаровать так, чтобы они были одновременными и при этом не препятствующими друг другу. Для такой координации нужно уметь точно контролировать роботов, т. е., нужен алгоритм калибровки. При этом, в силу простоты роботов, желательно, чтобы алгоритм также был прост.

\specialsection{Постановка задачи}

Есть 2 системы координат - система координат доски и система координат робота. Начало координат в СК доски - центр клетки a1, ось $Ox$ направлена вдоль букв, ось $Oy$ - вдоль цифр. 
В СК робота ось $Ox$ направлена вперед от робота, $Oz$ вертикально вверх, $Oy$ - так, чтобы $(Ox, Oy, Oz)$ образовывали правую тройку.
Все координаты считаются в миллиметрах.

Есть возможность приводить робота в некоторые точки на поверхности доски и в этот момент замерять координаты рабочего устройства в обеих СК.
Требутется:
 а) найти способ сбора координат точек;
 б) найти алгоритм, который по полученным координатам вычисляет матрицу перехода из СК доски в СК робота;
 в) оценить полученное решение.

\pagebreak

\section{Решение задачи о поиске матрицы перехода}

Пусть $n$ точек в СК доски задаются как векторы $\boldmath{x}_i, i=1,\dots,n$, а проекции соответствующие им векторы в СК робота на $Oxy$ - $\boldmath{y}_i, i=1,\dots,n$.

\subsection{Существующий алгоритм}
Перевод из одной СК в другую представим как последовательные поворот (с матрицей поворота $R$) и параллельный перенос (на вектор $\boldmath{p}$):

\begin{equation}
    \boldmath{y}_i = R\boldmath{x}_i + \boldmath{p}
    \label{eq:transformation}
\end{equation}
    
Так как значения векторов измеряются с погрешностью, для найденных $R, \boldmath{p}$ можно определить среднеквадратичное отклонение:

\begin{equation}
    \dfrac{1}{n}\sum\limits_{i=1}^n ||R\boldmath{x}_i + \boldmath{p} - \boldmath{y}_i||_2^2
    \label{eq:loss}
\end{equation}

Для поиска $R, \boldmath{p}$, минимизирующих \ref{eq:loss} используется алгоритм Кабша \cite{Kabsch}.

\subsection{Геометрическое решение}

Пусть для точек с номерами $i, j$ соответсвующие им $\boldmath{y}_i, \boldmath{y}_j$ неколлинеарны, то из геометрических соображений легко найти точку $Q_{ij}$ в СК доски, соответствующую базе робота. Из-за погрешностей измерения каждая пара $i, j$ будет давать свою $Q_ij$, поэтому за координаты центра $\hat Q$ будет взято среднее по $Q_{ij}$ (вдоль каждой из осей).

Поворот $\phi_{i}$ можно получить из угла $\boldmath{y}_i$ и $\boldmath{x}_i - \hat Q$. $\hat\phi$ - среднее по $\phi_i$.

\section{Оценка решений}

\subsection{Сбор данных}

Выбирается несколько позиций на доске, затем через программный интерфейс роботу посылаются команды, вследствие которых он сдвигается по одной из осей на некоторое расстояние (10 мм, 1 мм или 0.1 мм), пока его рабочее устройство не достигнет центра соответствующей ячейки, после чего считываются внутренние координаты робота. Процесс повторяется для нескольких точек на доске. Пример измерения 5 точек представлен в первых 2 столбцах таблицы 1. 

\subsection{Оценка решения}

Оценка качества производится методом складного ножа - для каждой $i$-й точки алгоритм ищет $R_{-i}$ и $p_{-i}$ по всем точкам кроме $i$-й, по \ref{eq:transformation} строится 
$\tilde{\boldmath{y}}_i$ и сравнивается с $y_i$. Ошибкой считается расстояние $||\tilde{\boldmath{y}}_i - y_i||_2$.

Результаты ошибок для алгоритма Кабша видны в таблице 1. В силу того, что алгоритм Кабша дает решение с минимальным среднеквадратичным отклонением на обучающем наборе,
проверка ошибки для геометрического метода не осуществлялась. Однако геометрический метод полезен тем, что он дает сразу множество $Q_{jk}$ и $\phi_j$, поэтому для него 
по методу складного ножа ищется разброс координат основания робота - среднеквадратичное отклонение выборки $\{\hat Q - Q_{jk}\}_{j,k}^{j\neq i, k\neq i}$.
\begin{center}
    \begin{longtable}{|p{2cm}|p{5cm}|p{5cm}|p{3cm}|}
    \caption{Измерение 5 позиций на доске и оценка ошибки методом складного ножа}\\
    \hline
    Позиция на доске & Замеренное положение, мм & Предсказанное положение, мм & Расстояние (ошибка), мм\\
    \hline
    a1&(178.2, 93.6)&(181.1, 95.2)&3.22\\\hline h1&(180.6, -104.8)&(186.2, -106.3)&5.83\\\hline g5&(299.9, -75.0)&(295.7, -74.7)&4.26\\\hline c5&(297.1, 40.8)&(294.0, 39.4)&3.41\\\hline d3&(237.1, 9.8)&(238.5, 10.0)&1.42\\\hline 
    
    \end{longtable}
\end{center}

\begin{center}
    \begin{longtable}{|p{2cm}|p{5cm}|}
    \caption{Среднеквадратичное отклонения при применении геометрического метода}\\
    \hline
    Позиция на доске & Разброс координат основаия робота, мм & Ошибка при применении алгоритма Кабша\\
    \hline
    a1&12.9&3.22\\\hline h1&15.6&5.83\\\hline g5&14.1&4.26\\\hline c5&7.3&3.41\\\hline d3&5.6&1.42\\\hline
    \end{longtable}
\end{center}

\subsection{Анализ результатов}

Погрешность в 3-5 миллиметров, с учетом того, что крышка наименьшей фигуры (пешки) представляет собой квадрат со стороной примерно 1.5 см, является слишком большой. При этом из таблицы 2 видна скореллированность ошибки при применении алгоритма Кабша и разброс координат основания робота, т. е. можно сказать, что погрешность в значительной степени порождается слишком грубым измерением и, если использовать более точный метод получения координат, то алгоритм Кабша даст достаточно хорошие результаты.

\pagebreak

\specialsection{Заключение}
Использованный алгоритм Кабша позволяет найти решение поставленной задачи, однако основной проблемой является низкая точность измерения. Одним из предполагаемых решений может послужить использование eye-in-hand - камеры, прикрепленной рядом с рабочим устройством робота и позволяющей более точно находить его положение в системе координат доски.

\begin{thebibliography}{1}
\bibitem{Kabsch} Kabsch W.\flqq A solution for the best rotation to relate two sets of vectors\frqq Acta Cryst, 1976, A32, 922-923
\end{thebibliography}
\end{document}